\section{Background}
\label{sec:background}

Erasure coding is a method of data protection by which data is 
broken into packets. These packets are then expanded and encoded 
with redundant data pieces so that the original data is still preserved 
in case of packet erasure or outages. In the context of communication 
networks, packet erasure coding \cite{WalrandParekh2017} is used in the case of multicasting, 
where it is infeasible for the source node to track acknowledgements from 
all the destination nodes. As such, this scheme is designed to be able to 
recover from the “erasures”, or dropping of packets as packets propagate 
through the network. This paper hopes to discuss and evaluate the effectiveness of packet
erasure coding in the case of unicasting within an unreliable network.

