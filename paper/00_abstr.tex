%%%%%%%%%%%%%%%%%%%%%%%%%%%%%%%%%%%%%%%%%%%%%%%%%%%%%%%%%%%%%%%%%%%%%%
%     File: ExtendedAbstract_abstr.tex                               %
%     Tex Master: ExtendedAbstract.tex                               %
%                                                                    %
%     Author: Andre Calado Marta                                     %
%     Last modified : 2 Dez 2011                                     %
%%%%%%%%%%%%%%%%%%%%%%%%%%%%%%%%%%%%%%%%%%%%%%%%%%%%%%%%%%%%%%%%%%%%%%
% The abstract of should have less than 500 words.
% The keywords should be typed here (three to five keywords).
%%%%%%%%%%%%%%%%%%%%%%%%%%%%%%%%%%%%%%%%%%%%%%%%%%%%%%%%%%%%%%%%%%%%%%

%%
%% Abstract
%%
\begin{abstract}
This report explores the efficacy of packet erasure coding in unreliable network environments. Packet erasure coding divides data into packets, encodes them with redundancy to preserve data integrity amidst packet loss, and is particularly vital when networks are unreliable. The study aims to find the optimal redundancy rate for successful packet decoding at end nodes based on the packet drop rates at the routers. Leveraging Docker for network simulation, controlled experiments vary redundancy factors and packet drop rates across diverse network topologies. The simulation results give insight into how we can adjust the redundancy factor in order to optimise decoding success rates based network reliability. 
    \\
    %%
    %% Keywords (max 5)
    %%
    \noindent{{\bf Keywords:}} Packet Erasure Codes, Network Simulation \\
    \noindent{{\bf https://github.com/minghancmh/network-emulation}} \\
    
    \end{abstract}
    
    